\chapter{Математическая основа гауссовского сплаттинга}
\label{cha:analysis}

В этом разделе излагаются ключевые математические идеи, лежащие в основе метода 3D-гауссовского сплаттинга. Подход строится на представлении сцены как параметризованного семейства гауссиан в трёхмерном пространстве, что позволяет заменить традиционные методы 3D-реконструкции (NeRF, полигональные модели) более гибким и дифференцируемым способом визуализации.

\section{Параметризация гауссовых сплаттов}

Каждый элемент сцены моделируется гауссианой с параметрами:

\begin{itemize}
  \item $\mu \in \mathbb{R}^3$ — центр гауссианы в мировом пространстве;
  \item $\Sigma \in \mathbb{R}^{3 \times 3}$ — ковариационная матрица, определяющая форму и ориентацию распределения;
  \item $C \in \mathbb{R}^3$ — вектор цветовых компонент (RGB);
  \item $\alpha \in [0, 1]$ — непрозрачность.
\end{itemize}

Плотность в пространстве выражается через функцию:
\[
f(\mathbf{x}) = C \cdot \exp\left(-\frac{1}{2} (\mathbf{x} - \mu)^T \Sigma^{-1} (\mathbf{x} - \mu)\right)
\]
Эта формула задаёт цветовое распределение, которое проецируется на изображение при рендеринге.

\section{Преобразования и проекция}

\subsection{Мировое и видовое пространство}

В контексте компьютерной графики принято различать два основных пространства координат:

\begin{itemize}
  \item \textbf{Мировое пространство} (\textit{world space}) — глобальная система координат сцены, в которой задаются положения всех объектов, включая камеры и источники света. Каждая точка имеет координаты $\mathbf{x} \in \mathbb{R}^3$.
  \item \textbf{Видовое пространство} (\textit{view space} или \textit{camera space}) — система координат, связанная с конкретной виртуальной камерой. Камера помещается в начало координат, ось $z$ направлена в сторону взгляда, а оси $x$ и $y$ — по горизонтали и вертикали изображения.
\end{itemize}

Переход из мирового пространства в видовое осуществляется матрицей $W \in \mathbb{R}^{4 \times 4}$:
\[
\mathbf{x}' = W \cdot \mathbf{x}
\]
Ковариационная матрица преобразуется аналогично:
\[
\Sigma' = U \Sigma U^T
\]
где $U$ — верхний левый блок якобиана проекции, отражающий локальное искажение гауссианы при проецировании на экран.

\subsection{Проецирование на изображение}

Для получения двумерной проекции используется матрица камеры:
\[
\mathbf{p} = \Pi \cdot \mathbf{x}'
\]
где $\Pi$ — матрица перспективного (или ортографического) проецирования.

\section{Градиенты для оптимизации}

\subsection{Частные производные и правило цепочки}

Оптимизация параметров гауссиан осуществляется методом градиентного спуска. При этом ключевую роль играют частные производные по параметрам масштабирования и поворота. В частности, производные преобразованной ковариационной матрицы $\Sigma'$ вычисляются согласно правилу цепочки:

\[
\frac{d\Sigma'}{ds} = \frac{d\Sigma'}{d\Sigma} \cdot \frac{d\Sigma}{ds}, \qquad
\frac{d\Sigma'}{dq} = \frac{d\Sigma'}{d\Sigma} \cdot \frac{d\Sigma}{dq}
\]

Частная производная по элементу $\Sigma_{ij}$ имеет следующий аналитический вид:

\[
\frac{\partial \Sigma'}{\partial \Sigma_{ij}} =
\begin{pmatrix}
U_{1i} U_{1j} & U_{1i} U_{2j} \\
U_{2i} U_{1j} & U_{2i} U_{2j}
\end{pmatrix}
\]

где $U$ — матрица проекции в экранное пространство.

\subsection{Градиенты по кватерниону поворота}

Матрица поворота $R(q)$, соответствующая кватерниону $q = (q_r, q_i, q_j, q_k)$, определяется следующим выражением:

\[
R(q) = 2
\begin{pmatrix}
\frac{1}{2} - (q_j^2 + q_k^2) & q_i q_j - q_r q_k & q_i q_k + q_r q_j \\
q_i q_j + q_r q_k & \frac{1}{2} - (q_i^2 + q_k^2) & q_j q_k - q_r q_i \\
q_i q_k - q_r q_j & q_j q_k + q_r q_i & \frac{1}{2} - (q_i^2 + q_j^2)
\end{pmatrix}
\]

Эта форма обеспечивает непрерывное и дифференцируемое задание поворота в трёхмерном пространстве.

\subsection{Градиенты по масштабу}

Дифференцирование по компоненте масштабирования $s_k$ даёт:

\[
\frac{\partial M_{ij}}{\partial s_k} = 
\begin{cases}
R_{ik}, & j = k \\
0, & \text{иначе}
\end{cases}
\]

где $M = R \cdot S$ — результирующая матрица линейного преобразования, представляющая собой произведение матрицы поворота $R$ и диагональной матрицы масштабирования $S$.

\section{Алгоритм оптимизации}

Ниже приведён псевдокод алгоритма оптимизации гауссиан, который включает инициализацию параметров, итеративный рендеринг и вычисление потерь, а также динамическое управление плотностью и качеством гауссиан. Основные шаги:

\begin{itemize}
    \item Инициализация позиций, ковариаций, цветов и прозрачностей на основе исходных данных SfM.
    \item Итеративное обновление параметров с помощью градиентного спуска (Adam), минимизирующего функцию потерь между текущим и целевым изображениями.
    \item Периодическая проверка и корректировка гауссиан: удаление слабых или слишком больших, а также разбиение или клонирование для улучшения детализации.
\end{itemize}

\begin{lstlisting}[style=pseudocode,caption={Gaussian Splatting Optimization Algorithm}]
procedure OptimizeGaussians()
    Initialize M <- initial positions from SfM
    Initialize S, C, A <- initial covariances, colors, opacities
    while not converged do
        V, I_target <- SampleTrainingView()
        I <- Render(M, S, C, A, V)
        L <- ComputeLoss(I, I_target)
        (M, S, C, A) <- AdamStep(grad L)
        if IsRefinementIteration() then
            for each Gaussian (mu, Sigma, C, alpha) do
                if alpha < eps or IsTooLarge(Sigma) then
                    RemoveGaussian()
                else if Norm(grad L) > tau_p then
                    if ScaleNorm(Sigma) > tau_S then
                        SplitGaussian(mu, Sigma, C, alpha)
                    else
                        CloneGaussian(mu, Sigma, C, alpha)
\end{lstlisting}
