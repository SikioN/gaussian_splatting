\section{Связанные работы}
Мы сначала кратко обозреваем традиционную реконструкцию, затем обсуждаем методы рендеринга на основе точек и радиусных полей, рассматривая их сходство; радиусные поля — это обширная область, поэтому мы фокусируемся только на непосредственно связанных работах. 
Для полного охвата области, пожалуйста, см. отличные недавние обзоры~\cite{tewari2022advances,xie2022neural}.
\subsection{Традиционная реконструкция сцен и рендеринг}
Первые подходы к синтезу новых видов были основаны на световых полях, сначала плотно семплированных~\cite{gortler1996lumigraph,levoy1996light}, а затем позволяющих неструктурированный захват~\cite{buehler2001unstructured}. Появление метода Structure-from-Motion (SfM)~\cite{snavely2006photo} открыло новую область, где коллекция фотографий могла использоваться для синтеза новых видов. SfM оценивает разреженное облако точек во время калибровки камеры, которое изначально использовалось для простой визуализации 3D пространства. Последующие многовидовые стерео методы (MVS) привели к появлению впечатляющих алгоритмов полной 3D реконструкции за прошедшие годы~\cite{goesele2007multi}, что позволило разработать несколько алгоритмов синтеза видов \cite{eisemann2008floating, chaurasia2013depth, hedman2018deep,kopanas21}.
Все эти методы \emph{репроектируют} и \emph{смешивают} входные изображения в камеру нового ракурса, используя геометрию для управления этой репроекцией. Эти методы давали отличные результаты во многих случаях, но обычно не могут полностью восстановиться после нереконструированных областей или из-за «чрезмерной реконструкции», когда MVS создает несуществующую геометрию.
Недавние алгоритмы нейронного рендеринга ~\cite{tewari2022advances} значительно уменьшают такие артефакты и избегают огромной стоимости хранения всех входных изображений на GPU, превосходя эти методы по большинству параметров.
\subsection{Нейронный рендеринг и радиусные поля}
Глубокие методы обучения были внедрены на ранних этапах для синтеза новых видов ~\cite{zhou2016view, flynn2016deepstereo}; сверточные нейронные сети (CNN) использовались для оценки весов смешивания~\cite{hedman2018deep}, или для решений в текстурном пространстве \cite{thies2019deferred,riegler2020free}. Использование геометрии, основанной на MVS, является основным недостатком большинства этих методов; кроме того, использование CNN для окончательного рендеринга \ADDITION{часто} приводит \REMOVAL{к частому} временному мерцанию.
Объемные представления для синтеза новых видов были инициированы Soft3D~\cite{penner2017soft}; глубокие методы обучения, связанные с объемным трассированием лучей, были предложены позднее~\cite{sitzmann2019deepvoxels,henzler2019escaping}, основываясь на непрерывном дифференцируемом поле плотности для представления геометрии. Рендеринг с использованием объемного трассирования лучей имеет значительную стоимость из-за большого количества выборок, необходимых для запроса объема. Нейронные радиусные поля (NeRFs)~\cite{mildenhall2020nerf} ввели важную выборку и позиционное кодирование для повышения качества, но использовали большой многослойный перцептрон, что негативно влияло на скорость.
Успех NeRF привел к взрыву последующих методов, которые решают проблемы качества и скорости, часто вводя стратегии регуляризации; текущее состояние техники в качестве изображения для синтеза новых видов — это Mip-NeRF360~\cite{barron2022mipnerf360}. Хотя качество рендеринга выдающееся, время обучения и рендеринга остается чрезвычайно высоким; мы можем соответствовать или в некоторых случаях превзойти это качество, обеспечивая быстрое обучение и рендеринг в реальном времени.
Наиболее современные методы сосредоточились на более быстром обучении и/или рендеринге, главным образом за счет трех проектных решений: использования пространственных структур данных для хранения (нейронных) признаков, которые затем интерполируются во время объемного трассирования лучей, различных кодировок и емкости MLP. Такие методы включают
различные варианты дискретизации пространства~\cite{nglod-cvpr2021,plenoxels,yu2021plenoctrees,hedman2021snerg,chen2022mobilenerf,garbin2021fastnerf,Reiser2021ICCV,tensorf-eccv2022,wu2022snisr},
кодовые книги~\cite{takikawa2022variable},
и кодировки, такие как хеш-таблицы~\cite{mueller2022instant}, что позволяет использовать меньший MLP или полностью отказаться от нейронных сетей \cite{plenoxels,dvgo-cvpr2022}. %
Наиболее примечательными из этих методов являются InstantNGP~\cite{mueller2022instant}, который использует хеш-сетку и сетку занятости для ускорения вычислений и меньший MLP для представления плотности и внешнего вида;
и Plenoxels~\cite{plenoxels}, которые используют разреженную воксельную сетку для интерполяции непрерывного поля плотности и могут полностью обойтись без нейронных сетей. \CORRECTION{Они оба заменяют MLP, используемый для представления направленных эффектов, гораздо более быстрыми сферическими гармониками.}
{Оба они полагаются на сферические гармоники: первый для прямого представления направленных эффектов, второй для кодирования своих входных данных в цветовую сеть.}
Хотя оба предоставляют выдающиеся результаты, эти методы все еще могут испытывать трудности с эффективным представлением пустого пространства, отчасти зависящие от типа сцены/захвата. Кроме того, качество изображения в значительной степени ограничено выбором структурированных сеток, используемых для ускорения, а скорость рендеринга ограничивается необходимостью запрашивать множество выборок для данного шага трассировки лучей. Неупорядоченные, явные, дружественные к GPU 3D Гауссианы, которые мы используем, обеспечивают более быструю скорость рендеринга и лучшее качество \emph{без} нейронных компонентов.
\subsection{Рендеринг на основе точек и радиусные поля}
Методы на основе точек эффективно рендерят отсоединенные и неструктурированные геометрические образцы (например, облака точек)~\cite{gross2011point}. 
В самой простой форме рендеринг точек выборки \cite{Grossman1998PointSR} растеризует неструктурированный набор точек с фиксированным размером, для которого может быть использована нативная поддержка типов точек графических API \cite{SAINZ2004869} или параллельная программная растеризация на GPU \cite{10.1145/3543863, laine2011high}. Будучи верным исходным данным, рендеринг точек выборки страдает от дыр, вызывает алиасинг и является строго разрывным. Семинальная работа по качественному рендерингу на основе точек решает эти проблемы путем "сплаттинга" точечных примитивов с протяженностью больше пикселя, например, круглых или эллиптических дисков, эллипсоидов или сурфелей \cite{10.1145/383259.383300, 10.5555/2386366.2386369, Ren2002ObjectSE, 10.1145/344779.344936}.
Появился недавний интерес к \emph{дифференцируемым} методам рендеринга на основе точек~\cite{yifan19,wiles2020synsin}. Точки были дополнены нейронными признаками и отрендерены с использованием CNN~\cite{ruckert22,aliev20}, что привело к быстрому или даже реальному времени синтезу видов; однако они все еще зависят от MVS для начальной геометрии, и таким образом наследуют его артефакты, особенно пере- или недо-реконструкцию в сложных случаях, таких \CORRECTION{}{как} бесособые/блестящие области или тонкие структуры. 
Альфа-смешивание на основе точек и стильный рендеринг NeRF имеют, по сути, одну и ту же модель формирования изображения.
В частности, цвет $C$ задается объемным рендерингом вдоль луча:
\begin{equation}
    \label{eq:nerf-rendering-quadrature}
    C = \sum_{i=1}^N T_i(1-\exp(-\sigma_i\delta_i))\mathbf{c}_i \hspace{0.5em} \text{ with } \hspace{0.5em} T_i = \exp\left(-\sum_{j=1}^{i-1}\sigma_j\delta_j\right),
\end{equation}
где выборки плотности $\sigma$, пропускания $T$ и цвета $\mathbf{c}$ берутся вдоль луча с интервалами $\delta_i$.
Это можно переписать как
\begin{equation}
    \label{eq:nerf-rendering-quadrature2}
    C = \sum_{i=1}^N T_i\alpha_i\mathbf{c}_i,
\end{equation}
с
\begin{equation*}
    \alpha_i = (1-\exp(-\sigma_i\delta_i)) 
    \hspace{0.5em} \text{and} \hspace{0.5em}
    T_i = \prod_{j=1}^{i-1}(1-\alpha_i).
\end{equation*}
Типичный нейронный подход на основе точек (например,~\cite{kopanas21,kopanas22}) вычисляет цвет $C$ пикселя путем смешивания $\mathcal{N}$ упорядоченных точек, перекрывающих пиксель:
\begin{equation}
    \label{eq:front-to-back}
    C = \sum_{i \in \mathcal{N}}
    c_{i}\alpha_{i}
    \prod_{j=1}^{i-1}(1-\alpha_{j}),
\end{equation}
где $\mathbf{c}_i$ — это цвет каждой точки и $\alpha_i$ задается путем оценки 2D Гауссиана с ковариацией $\Sigma$ ~\cite{yifan19}, умноженного на обучаемую по-точечную непрозрачность. 
Из уравнений~\ref{eq:nerf-rendering-quadrature2} и~\ref{eq:front-to-back} мы можем ясно видеть, что модель формирования изображения одна и та же. Однако алгоритм рендеринга очень различен. NeRF представляет собой непрерывное представление, неявно представляющее пустое/занятое пространство; требуется дорогая случайная выборка для нахождения выборок в уравнении~\ref{eq:nerf-rendering-quadrature2} с последующим шумом и вычислительными затратами. Напротив, точки представляют собой неструктурированное, дискретное представление, достаточно гибкое для создания, уничтожения и перемещения геометрии, аналогично NeRF. Это достигается путем оптимизации непрозрачности и положений, как показано в предыдущих работах~\cite{kopanas21}, избегая недостатков полноценного объемного представления. 
Pulsar~\cite{Lassner_2021_CVPR} достигает быстрой \emph{сферической} растеризации, которая вдохновила нашу тайловую систему и алгоритм сортировки рендеринга. Однако, учитывая анализ выше, мы хотим сохранять (приближенное) обычное $\alpha$-смешивание на отсортированных сплаттах для получения преимуществ объемных представлений: наша растеризация учитывает порядок видимости в отличие от их метода, не зависящего от порядка. Кроме того,
\CORRECTION{}{мы} обратно распространяем градиенты на все сплатты в пикселе и растеризуем анизотропные сплатты.
Все эти элементы способствуют высокому визуальному качеству наших результатов (см. раздел~\ref{sec:ablations}).
Кроме того,
упомянутые выше предыдущие методы
также используют CNN для рендеринга, что приводит к временной нестабильности.
\CORRECTION{}{Тем не менее, скорость рендеринга Pulsar~\cite{Lassner_2021_CVPR} и ADOP~\cite{ruckert22} послужила мотивацией для разработки нашего быстрого решения рендеринга.}
Хотя фокусируясь на зеркальных эффектах, диффузный трек рендеринга на основе точек Neural Point Catacaustics~\cite{kopanas22} преодолевает эту временную нестабильность с помощью MLP, но все же требует геометрию MVS на входе.
Самый последний метод~\cite{zhang2022} в этой категории не требует MVS и также использует SH для направлений; однако он может обрабатывать только сцены одного объекта и требует маски для инициализации. Хотя он быстр для малых разрешений и небольшого количества точек, неясно, как он сможет масштабироваться до сцен типичных наборов данных~\cite{knapitsch2017tanks,hedman2018deep,barron2022mipnerf360}.
Мы используем 3D Гауссианы для более гибкого представления сцены, избегая необходимости в геометрии MVS и достигая рендеринга в реальном времени благодаря нашему тайловому алгоритму рендеринга для спроецированных Гауссиан.
Недавний подход~\cite{xu2022point} использует точки для представления радиусного поля с помощью подхода функции радиального базиса. Они применяют методы прореживания и уплотнения точек во время оптимизации, но используют объемное трассирование лучей и не могут достичь скоростей реального времени.%
В области захвата человеческих действий 3D Гауссианы использовались для представления захваченных человеческих тел~\cite{stoll2011fast,rhodin2015versatile}; совсем недавно \ADDITION{они использовались с объемным трассированием лучей для задач зрения~\cite{voge}.} \CORRECTION{нейронные}{Нейронные} объемные примитивы были предложены в подобном контексте~\cite{lombardi2021mixture}. \CORRECTION{Хотя мы также используем}{Хотя эти методы вдохновили выбор} 3D Гауссианы в качестве нашего представления сцены, \CORRECTION{эти методы}{они} сосредотачиваются на конкретном случае реконструкции и рендеринга одного изолированного объекта (тела человека или лица), что приводит к сценам с малой глубиной сложности. В противоположность этому, наша оптимизация \emph{анизотропной} ковариации, чередующаяся оптимизация/контроль плотности и эффективная сортировка по глубине для рендеринга позволяют нам обрабатывать полные, сложные сцены, включая фон, как внутри помещений, так и на открытом воздухе, и с большой глубиной сложности.