\Conclusion

В ходе работы:

\begin{itemize}
    \item Изучены основы гауссовского сплаттинга (см. раздел~\ref{cha:analysis}).
    \item Реализована текстово-управляемая генерация 3D-сцен с использованием CLIP (см. раздел~\ref{cha:text_to_3d}).
    \item Построена 3D-модель подсолнухов Ван Гога (см. рис.~\ref{fig:sunflowers_3d}), подтверждающая корректность реализации.
\end{itemize}

\section*{Применение в робототехнике}

Разработанный метод может быть использован в следующих задачах:

\begin{itemize}
    \item \textbf{Навигация и SLAM} — построение детализированных 3D-карт и оптимизация маршрутов.
    \item \textbf{Обнаружение объектов} — текстово-управляемое распознавание и локализация в сцене.
    \item \textbf{Симуляция и обучение} — генерация реалистичных сцен для обучения агентов.
    \item \textbf{Обработка LiDAR-данных} — сглаживание, реконструкция и представление облаков точек в виде гауссиан.
\end{itemize}

\section*{Преимущества и ограничения}

\textbf{Преимущества:} высокая скорость рендеринга, поддержка динамических сцен, интеграция с языковыми моделями.\\
\textbf{Ограничения:} высокая вычислительная сложность, потребность в тонкой настройке гиперпараметров и производительном GPU.

\section*{Выводы}

Разработана и реализована система текстово-управляемой 3D-реконструкции на основе гауссовского сплаттинга. В перспективе возможны улучшения, направленные на:

\begin{itemize}
    \item оптимизацию под мобильные и встроенные устройства;
    \item интеграцию с физическими симуляторами;
    \item реализацию потоковой обработки в реальном времени.
\end{itemize}

\vspace{1em}
\section*{Репозиторий с исходными материалами}

Все исходные материалы, включая код, модели и примеры, находятся в открытом доступе на GitHub по адресу:
\begin{center}
  \href{https://github.com/SikioN/gaussian_splatting}{\texttt{gaussian\_splatting\_semantic}}
\end{center}
