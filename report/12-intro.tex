\Introduction


Цель работы — исследование и применение 3D-гауссовского сплаттинга для создания качественных 3D-реконструкций с использованием методов машинного обучения. Для этого необходимо:

\begin{itemize}
    \item Изучить статью по 3D-сплаттингу и текущее состояние исследований;
    \item Рассмотреть математическую модель: гауссианы и рендеринг;
    \item Освоить инструменты Yandex DataSphere для запуска на GPU;
    \item Реализовать реконструкцию с использованием оригинального репозитория \texttt{gaussian-splatting}.
\end{itemize}

\bigskip

\Abbrev{Gaussian Splatting}{представление 3D-сцен через облака гауссиан с параметрами цвета, положения и формы}

\Abbrev{NeRF}{нейросетевые поля излучения, используемые для 3D-реконструкции}

\Define{Гауссовский сплаттинг}{Метод визуализации 3D-сцены через гауссианы с использованием дифференцируемого растеризатора}

Полученные в ходе работы навыки имеют непосредственное применение в задачах робототехники:
\begin{itemize}
    \item \textbf{Навигации} — построение карт по видеопотоку;
    \item \textbf{Распознавания} — высокая детализация сцены;
    \item \textbf{Симуляций} — быстрый рендеринг для обучения агентов;
    \item \textbf{Обработки LiDAR} — представление облаков точек через гауссианы.
\end{itemize}

Yandex DataSphere обеспечивает доступ к ускоренным вычислениям и упрощает запуск ресурсоёмких задач, что особенно важно для работы в реальном времени.
