\section{Введение}

Сетки и точки являются наиболее распространенными представлениями 3D-сцен, поскольку они явные и хорошо подходят для быстрой растеризации на базе GPU/CUDA. 
В отличие от них, современные методы Нейронных Радиусных Полей (NeRF) основываются на непрерывных представлениях сцен, обычно оптимизируя многослойный перцептрон (MLP) с использованием объемного трассирования лучей для синтеза новых видов захваченных сцен. Аналогично, самые эффективные на данный момент решения радиусных полей строятся на непрерывных представлениях, интерполируя значения, хранящиеся, например, в воксельных~\cite{plenoxels} или хеш-сетках~\cite{mueller2022instant}, или точках~\cite{xu2022point}.
Хотя непрерывная природа этих методов способствует оптимизации, стохастическая выборка, необходимая для рендеринга, является дорогостоящей и может привести к шуму.
Мы представляем новый подход, который сочетает в себе лучшее из обоих миров: наши 3D Гауссианы позволяют проводить оптимизацию с качеством визуализации на уровне современного уровня (SOTA) и конкурентным временем обучения,
в то время как наше решение на основе тайлового сплаттинга обеспечивает рендеринг в реальном времени с SOTA качеством для разрешения 1080p на нескольких ранее опубликованных наборах данных~\cite{knapitsch2017tanks,hedman2018deep,barron2022mipnerf360} (см. Рис.~\ref{fig:teaser}).


Наша цель — позволить рендеринг в реальном времени для сцен, захваченных с помощью множества фотографий, и создать представления с временем оптимизации, таким же быстрым, как у самых эффективных предыдущих методов для типичных реальных сцен.
Недавние методы достигают быстрого обучения~\cite{plenoxels,mueller2022instant}, но испытывают трудности в достижении качества визуализации, получаемого текущими SOTA NeRF методами, т. е., Mip-NeRF360~\cite{barron2022mipnerf360}, который требует до 48 часов времени обучения. Быстрые, но менее качественные методы радиусных полей могут достигать интерактивного времени рендеринга в зависимости от сцены (10–15 кадров в секунду), но не достигают рендеринга в реальном времени \CORRECTION{($\geq$ 30~fps)}{} при высоком разрешении.

Наше решение строится на трех основных компонентах. Мы сначала вводим \emph{3D Гауссианы} как гибкое и выразительное представление сцены.
Мы начинаем с того же входного сигнала, что и предыдущие методы, похожие на NeRF, т.е., камеры, откалиброванные с помощью Structure-from-Motion (SfM) \cite{snavely2006photo}, и инициализируем набор 3D Гауссиан с использованием разреженного облака точек, которое создается бесплатно как часть процесса SfM. В отличие от большинства решений на основе точек, которые требуют данных Многовидового Стерео (MVS)~\cite{aliev20,kopanas21,ruckert22}, мы достигаем высококачественных результатов только с использованием точек SfM на входе. Отметим, что для NeRF-синтетического набора данных наш метод достигает высокого качества даже при случайной инициализации.
Мы показываем, что 3D Гауссианы являются отличным выбором, поскольку это дифференцируемое объемное представление, но их также можно очень эффективно растеризовать, проецируя их в 2D и применяя стандартное $\alpha$-смешивание, используя ту же модель формирования изображения, что и NeRF.
Второй компонент нашего метода заключается в оптимизации свойств 3D Гауссиан — 3D позиции, непрозрачности $\alpha$, анизотропной ковариации и коэффициентов сферических гармоник (SH) — чередующихся с шагами адаптивного контроля плотности, где мы добавляем и иногда удаляем 3D Гауссианы во время оптимизации. Процедура оптимизации создает достаточно компактное, неструктурированное и точное представление сцены (1–5 миллионов Гауссиан для всех протестированных сцен).
Третий и последний элемент нашего метода — это наше решение для рендеринга в реальном времени, которое использует быстрые алгоритмы сортировки GPU и вдохновлено тайловой растеризацией, следуя недавним работам~\cite{Lassner_2021_CVPR}. Однако благодаря нашему представлению 3D Гауссиан мы можем выполнять анизотропное сплатирование, учитывая порядок видимости — благодаря сортировке и $\alpha$-смешиванию — и обеспечивать быстрый и точный обратный проход, отслеживая прохождение стольких отсортированных сплатов, сколько необходимо.



\noindent
Подводя итог, мы предоставляем следующие вклады:

    
- Введение анизотропных 3D Гауссиан как высококачественного, неструктурированного представления радиусных полей.
    
- Метод оптимизации свойств 3D Гауссиан, чередующийся с адаптивным контролем плотности, создающий высококачественные представления для захваченных сцен.
    
- Быстрый, дифференцируемый подход к рендерингу для GPU, учитывающий видимость, позволяющий анизотропное сплатирование и быстрое обратное распространение для достижения высококачественного синтеза новых видов.


\noindent
Наши результаты на ранее опубликованных наборах данных показывают, что мы можем оптимизировать наши 3D Гауссианы из многовидовых захватов и достичь качества, равного или лучшего, чем у лучших предыдущих неявных методов радиусных полей. Мы также можем достичь скоростей обучения и качества, аналогичных самым быстрым методам, и, что важно, предоставить первый \emph{рендеринг в реальном времени} с высоким качеством для синтеза новых видов.