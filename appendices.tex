\begin{appendix}
    \section{Подробности вычисления градиентов}
    \label{sec:appa}
    Напомним, что $\Sigma$/$\Sigma'$ — это матрицы ковариации Гауссиан в мировом/видовом пространстве, $q$ — это вращение, а $s$ — масштабирование, $W$ — это преобразование просмотра, а $J$ — Якобиан аффинного приближения проективного преобразования.
    Мы можем применить цепное правило для нахождения производных по масштабированию и вращению:
    \begin{equation}
        \frac{d\Sigma'}{ds} = \frac{d\Sigma'}{d\Sigma}\frac{d\Sigma}{ds}
    \end{equation}
    и 
    \begin{equation}
        \frac{d\Sigma'}{dq} = \frac{d\Sigma'}{d\Sigma}\frac{d\Sigma}{dq}
    \end{equation}
    Упростив уравнение~\ref{eq:volume-render}, используя $U = JW$ и $\Sigma'$ как (симметричную) верхнюю левую $2\times2$ матрицу $U \Sigma U^T$, обозначая элементы матрицы индексами, мы можем найти частные производные
    $
        \frac{\partial \Sigma'}{\partial \Sigma_{ij}} = \left(\begin{smallmatrix}
            U_{1,i}U_{1,j} & U_{1,i} U_{2,j}\\
            U_{1,j} U_{2,i} & U_{2,i}U_{2,j}
        \end{smallmatrix}\right)
    $.
    Далее ищем производные $\frac{d\Sigma}{ds}$ и $\frac{d\Sigma}{dq}$. Поскольку $\Sigma = RSS^TR^T$, мы можем вычислить $M = RS$ и переписать $\Sigma = MM^T$. Таким образом, мы можем записать $\frac{d\Sigma}{ds} = \frac{d\Sigma}{dM} \frac{dM}{ds}$ и $\frac{d\Sigma}{dq} = \frac{d\Sigma}{dM} \frac{dM}{dq}$. Так как матрица ковариации $\Sigma$ (и её градиент) симметрична, общая первая часть компактно находится как $\frac{d\Sigma}{dM} = 2M^T$. Для масштабирования также имеем
    $	\frac{\partial M_{i,j}}{\partial s_k} =  \left\{\begin{array}{lr}
        R_{i,k} & \text{если j = k}\\
        0 & \text{иначе}
    \end{array}\right\}$.
    Чтобы вывести градиенты для вращения, вспомним преобразование единичного кватерниона $q$ с действительной частью $q_r$ и мнимыми частями $q_i, q_j, q_k$ в матрицу вращения $R$:
    \begin{equation}
        R(q) = 2\begin{pmatrix}
            \frac{1}{2} - (q_j^2 + q_k^2) & (q_i q_j - q_r q_k) & (q_i q_k + q_r q_j)\\
            (q_i q_j + q_r q_k) & \frac{1}{2} - (q_i^2 + q_k^2) & (q_j q_k - q_r q_i)\\
            (q_i q_k - q_r q_j) & (q_j q_k + q_r q_i) & \frac{1}{2} - (q_i^2 + q_j^2)
        \end{pmatrix}
        \label{eq:quat}
    \end{equation}
    В результате получаем следующие градиенты для компонентов $q$:
    \begin{equation}
        \begin{aligned}
            &\frac{\partial M}{\partial q_r} = 2 \left(\begin{smallmatrix}
                0 & -s_y q_k & s_z q_j\\
                s_x q_k & 0 & -s_z q_i\\
                -s_x q_j & s_y q_i & 0
            \end{smallmatrix}\right), 
            &\frac{\partial M}{\partial q_i} = 2\left(\begin{smallmatrix}
                0 & s_y q_j & s_z q_k\\
                s_x q_j & -2 s_y q_i & -s_z q_r\\
                s_x q_k & s_y q_r & -2 s_z q_i
            \end{smallmatrix}\right)
            \\
            &\frac{\partial M}{\partial q_j} = 2\left(\begin{smallmatrix}
                -2 s_x q_j & s_y q_i & s_z q_r\\
                s_x q_i & 0 & s_z q_k\\
                -s_x q_r & s_y q_k & -2s_z q_j
            \end{smallmatrix}\right),
            &\frac{\partial M}{\partial q_k} = 2\left(\begin{smallmatrix}
                -2 s_x q_k & -s_y q_r & s_z q_i\\
                s_x q_r & -2s_y q_k & s_z q_j\\
                s_x q_i & s_y q_j & 0
            \end{smallmatrix}\right)
        \end{aligned}
    \end{equation} 
Вывод градиентов для нормализации кватерниона тривиален.
    \section{Алгоритм оптимизации и увеличения плотности}
    Наши алгоритмы оптимизации и увеличения плотности суммированы в Алгоритме \ref{alg:optimization}.
    \begin{algorithm}[!h]
        \caption{Оптимизация и увеличение плотности\\
        $w$, $h$: ширина и высота обучающих изображений}
        \label{alg:optimization}
        \begin{algorithmic}
            \State $M \gets$ SfM Точки	\Comment{Позиции}
            \State $S, C, A \gets$ InitAttributes() \Comment{Ковариации, Цвета, Непрозрачности}
            \State $i \gets 0$	\Comment{Счетчик итераций}
            \While{не сходится}
            \State $V, \hat{I} \gets$ SampleTrainingView()	\Comment{Камера $V$ и Изображение}
            \State $I \gets$ Растеризовать($M$, $S$, $C$, $A$, $V$)	\Comment{Алг.~\ref{alg:rasterize}}
            \State $L \gets Loss(I, \hat{I}) $ \Comment{Потери}
            \State $M$, $S$, $C$, $A$ $\gets$ Adam($
abla L$) \Comment{Обратное распространение ошибки и шаг}
            \If{IsRefinementIteration($i$)}
            \ForAll{Гауссианы $(\mu, \Sigma, c, \alpha)$ $\textbf{в}$ $(M, S, C, A)$}
            \If{$\alpha < \epsilon$ или IsTooLarge($\mu, \Sigma)$}	\Comment{Обрезка}
            \State УдалитьГауссиану()	
            \EndIf
            \If{$
abla_p L > \tau_p$} \Comment{Увеличение плотности}
            \If{$\|S\| > \tau_S$}	\Comment{Пересоздание}
            \State РазделитьГауссиану($\mu, \Sigma, c, \alpha$)
            \Else								\Comment{Недостаточная реконструкция}
            \State КлонироватьГауссиану($\mu, \Sigma, c, \alpha$)
            \EndIf	
            \EndIf
            \EndFor		
            \EndIf
            \State $i \gets i+1$
            \EndWhile
        \end{algorithmic}
    \end{algorithm}
    \section{Подробности растеризатора}
    \label{app:raster}
    \paragraph{Сортировка.}
    Наш дизайн основан на предположении о большом количестве маленьких сплатов, и мы оптимизируем это путем однократной сортировки сплатов для каждого кадра с использованием поразрядной сортировки в начале.
    Мы разделяем экран на тайлы размером 16x16 пикселей (или бины). Мы создаем список сплатов для каждого тайла, создавая экземпляр каждого сплата в каждом перекрывающем его тайле 16$\times$16. Это приводит к умеренному увеличению числа Гауссиан для обработки, 
    что, однако, амортизируется более простым потоком управления и высокой параллельностью оптимизированной поразрядной сортировки GPU~\cite{merrill2010revisiting}.
    Мы назначаем ключ для каждого экземпляра сплата длиной до 64 бит, где младшие 32 бита кодируют его проекционную глубину, а старшие биты кодируют индекс перекрываемого тайла. Точная длина индекса зависит от того, сколько тайлов помещается в текущее разрешение. Сортировка по глубине таким образом решается напрямую для всех сплатов параллельно с помощью одной поразрядной сортировки. После сортировки мы можем эффективно создавать списки Гауссиан для обработки для каждого тайла, определяя начало и конец диапазонов в отсортированном массиве с одинаковым ID тайла. Это делается параллельно, запуская один поток для каждого 64-битного элемента массива для сравнения его старших 32 бит с двумя соседями. 
    По сравнению с \cite{Lassner_2021_CVPR}, наша растеризация полностью исключает последовательные этапы обработки примитивов и создает более компактные списки для обхода во время прямого прохода для каждого тайла.
    Мы показываем высокоуровневый обзор подхода к растеризации в Алгоритме~\ref{alg:rasterize}.
    \begin{algorithm}
        \caption{GPU программный растеризатор 3D Гауссиан\\
            $w$, $h$: ширина и высота изображения для растеризации\\
            $M$, $S$: средние значения и ковариации Гауссиан в мировом пространстве\\
            $C$, $A$: цвета и непрозрачности Гауссиан\\
            $V$: конфигурация камеры текущего вида}
        \label{alg:rasterize}
        \begin{algorithmic}
            \Function{Растеризовать}{$w$, $h$, $M$, $S$, $C$, $A$, $V$}
            \State ОбрезатьГауссианы($p$, $V$) \Comment{Отсечение пирамиды видимости}
            \State $M', S'$ $\gets$ ScreenspaceGaussians($M$, $S$, $V$) \Comment{Преобразование}
            \State $T$ $\gets$ СоздатьТайлы($w$, $h$)
            \State $L$, $K$ $\gets$ ДублироватьСКлючами($M'$, $T$) \Comment{Индексы и ключи}
            \State СортироватьПоКлючам($K$, $L$)							\Comment{Глобальная сортировка}
            \State $R$ $\gets$ ОпределитьДиапазоныТайлов($T$, $K$)
            \State $I \gets \mathbf{0}$ \Comment{Инициализация холста}
            \ForAll{Тайлы $t$ $\textbf{в}$ $I$}
            \ForAll{Пиксели $i$ $\textbf{в}$ $t$}
            \State $r \gets$ ПолучитьДиапазонТайла($R$, $t$)
            \State $I[i] \gets$ СмешатьПоПорядку($i$, $L$, $r$, $K$, $M'$, $S'$, $C$, $A$)
            \EndFor
            \EndFor
            \Return $I$
            \EndFunction
        \end{algorithmic}
    \end{algorithm}
    \ADDITION{\paragraph{Числовая стабильность.} Во время обратного прохода мы восстанавливаем промежуточные значения непрозрачности, необходимые для вычисления градиентов, многократно деля накопленную непрозрачность из прямого прохода на $\alpha$ каждой Гауссианы. Если реализовано наивно, этот процесс подвержен числовым нестабильностям (например, деление на 0). Чтобы решить эту проблему, как в прямом, так и в обратном проходе, мы пропускаем любые обновления смешивания с $\alpha < \epsilon$ (мы выбираем $\epsilon$ равным $\frac{1}{255}$) и также ограничиваем $\alpha$ сверху значением $0.99$. Наконец, \textbf{перед} тем как Гауссиана будет включена в прямой проход растеризации, мы вычисляем накопленную непрозрачность, если бы она была включена, и прекращаем фронтальное смешивание \textbf{до} того, как оно сможет превысить $0.9999$.}
    \section{Метрики ошибок для каждой сцены}
    \label{sec:appd}
    \ADDITION{Таблицы~\ref{tab:360_scene_ssim}--\ref{tab:ttdb_scene_lpips} содержат различные собранные метрики ошибок для нашей оценки всех рассматриваемых техник и реальных сцен. Мы приводим как скопированные числа Mip-NeRF360, так и те, которые были получены нами для создания изображений в статье; средние значения по полному набору данных Mip-NeRF360 составляют PSNR 27.58, SSIM 0.790 и LPIPS 0.240.}
    \begin{table}[H]
    \caption{SSIM метрики для сцен Mip-NeRF360. $\dagger$ скопировано из оригинальной статьи.}
    \scalebox{0.6}{
        \centering
        \begin{tabular}{l|ccccc|cccc}
        ~ & велосипед & цветы & сад & пень & холм с деревьями  & комната & стол & кухня & бонсай \\ \hline
        Plenoxels & 0.496 & 0.431 & 0.6063 & 0.523 & 0.509 & 0.8417 & 0.759 & 0.648 & 0.814 \\ 
        INGP-Base & 0.491 & 0.450 & 0.649 & 0.574 & 0.518  & 0.855 & 0.798 & 0.818 & 0.890 \\ 
        INGP-Big & 0.512 & 0.486 & 0.701 & 0.594 & 0.542  & 0.871 & 0.817 & 0.858 & 0.906 \\ 
        Mip-NeRF360$^\dagger$ & 0.685 & 0.583 & 0.813 & 0.744 & 0.632 & 0.913 & 0.894 & 0.920 & \textbf{0.941} \\ 
        Mip-NeRF360 & 0.685 & 0.584 & 0.809 & 0.745 & 0.631 & 0.910 & 0.892 & 0.917 & 0.938\\
        Ours-7k & 0.675 & 0.525 & 0.836 & 0.728 & 0.598  & 0.884 & 0.873 & 0.900 & 0.910 \\ 
        Ours-30k & \textbf{0.771} & \textbf{0.605} & \textbf{0.868} & \textbf{0.775} & \textbf{0.638} & \textbf{0.914} & \textbf{0.905} & \textbf{0.922} & 0.938 \\ 
    \end{tabular}
    }
    \label{tab:360_scene_ssim}
    \end{table}
    \begin{table}[H]
    \caption{PSNR метрики для сцен Mip-NeRF360. $\dagger$ скопировано из оригинальной статьи. }
    \scalebox{0.6}{
        \centering
        \centering
        \begin{tabular}{l|ccccc|cccc}
            ~ & велосипед & цветы & сад & пень & холм с деревьями  & комната & стол & кухня & бонсай \\ \hline
            Plenoxels & 21.912 & 20.097 & 23.4947 & 20.661 & 22.248 & 27.594 & 23.624 & 23.420 & 24.669 \\ 
            INGP-Base & 22.193 & 20.348 & 24.599 & 23.626 & 22.364  & 29.269 & 26.439 & 28.548 & 30.337 \\ 
            INGP-Big & 22.171 & 20.652 & 25.069 & 23.466 & 22.373  & 29.690 & 26.691 & 29.479 & 30.685 \\ 
            Mip-NeRF360$^\dagger$ & 24.37 & \textbf{21.73} & 26.98 & 26.40 & 22.87 & \textbf{31.63} & \textbf{29.55} & \textbf{32.23} & \textbf{33.46} \\
            Mip-NeRF360 & 24.305 & 21.649 & 26.875 & 26.175 & \textbf{22.929} & 31.467 & 29.447 & 31.989 & 33.397 \\
            Ours-7k & 23.604 & 20.515 & 26.245 & 25.709 & 22.085  & 28.139 & 26.705 & 28.546 & 28.850 \\ 
            Ours-30k & \textbf{25.246} & 21.520 & \textbf{27.410} & \textbf{26.550} & 22.490 & 30.632 & 28.700 & 30.317 & 31.980 \\ 
        \end{tabular}
    }
    \label{tab:360_scene_psnr}
\end{table}
    \begin{table}[H]
    \caption{LPIPS метрики для сцен Mip-NeRF360.  $\dagger$ скопировано из оригинальной статьи.}
    \scalebox{0.6}{
        \centering
    \begin{tabular}{l|ccccc|cccc}
    ~ & велосипед & цветы & сад & пень & холм с деревьями  & комната & стол & кухня & бонсай \\ \hline
    Plenoxels & 0.506 & 0.521 & 0.3864 & 0.503 & 0.540 & 0.4186 & 0.441 & 0.447 & 0.398 \\ 
    INGP-Base & 0.487 & 0.481 & 0.312 & 0.450 & 0.489  & 0.301 & 0.342 & 0.254 & 0.227 \\ 
    INGP-Big & 0.446 & 0.441 & 0.257 & 0.421 & 0.450  & 0.261 & 0.306 & 0.195 & 0.205 \\ 
    Mip-NeRF360$^\dagger$ & 0.301 & 0.344 & 0.170 & 0.261 &  0.339 & \textbf{0.211} & \textbf{0.204} & \textbf{0.127} & \textbf{0.176} \\ 
    Mip-NeRF360 & 0.305 & 0.346 & 0.171 & 0.265 & 0.347 & 0.213 & 0.207 & 0.128 & 0.179\\
    Ours-7k & 0.318 & 0.417 & 0.153 & 0.287 & 0.404  & 0.272 & 0.254 & 0.161 & 0.244 \\ 
    Ours-30k & \textbf{0.205} & \textbf{0.336} & \textbf{0.103} & \textbf{0.210} & \textbf{0.317} & 0.220 & \textbf{0.204} & 0.129 & 0.205 \\ 
\end{tabular}
    }
\end{table}
    \begin{table}[H]
    \caption{SSIM метрики для сцен Tanks\&Temples и Deep Blending. }
        \centering
    \begin{tabular}{l|cc|cc}
    ~ & Грузовик & Поезд & Доктор Джонсон & Игровая комната \\ \hline
    Plenoxels & 0.774 & 0.663 & 0.787 & 0.802 \\ 
    INGP-Base & 0.779 & 0.666 & 0.839 & 0.754 \\ 
    INGP-Big & 0.800 & 0.689 & 0.854 & 0.779 \\ 
    Mip-NeRF360 & 0.857 & 0.660 & \textbf{0.901} & 0.900 \\ 
    Ours-7k & 0.840 & 0.694 & 0.853 & 0.896 \\ 
    Ours-30k & \textbf{0.879} & \textbf{0.802} & 0.899 & \textbf{0.906} \\ 
\end{tabular}
\end{table}
    \begin{table}[H]
    \caption{PSNR метрики для сцен Tanks\&Temples и Deep Blending. }
        \centering
        \begin{tabular}{l|cc|cc}
            ~ & Грузовик & Поезд & Доктор Джонсон & Игровая комната \\ \hline
            Plenoxels & 23.221 & 18.927 & 23.142 & 22.980 \\ 
            INGP-Base & 23.260  & 20.170 & 27.750 & 19.483 \\ 
            INGP-Big & 23.383  & 20.456 & 28.257 & 21.665 \\ 
            Mip-NeRF360 & 24.912  & 19.523 & \textbf{29.140} & 29.657 \\ 
            Ours-7k & 23.506  & 18.892 & 26.306 & 29.245 \\ 
            Ours-30k & \textbf{25.187} & \textbf{21.097} & 28.766 & \textbf{30.044} \\ 
        \end{tabular}
\end{table}
    \begin{table}[H]
    \caption{LPIPS метрики для сцен Tanks\&Temples и Deep Blending. }
        \centering
    \begin{tabular}{l|cc|cc}
    ~ & Грузовик & Поезд & Доктор Джонсон & Игровая комната \\ \hline
    Plenoxels & 0.335 & 0.422 & 0.521 & 0.499 \\ 
    INGP-Base & 0.274 & 0.386 & 0.381 & 0.465 \\ 
    INGP-Big & 0.249 & 0.360 & 0.352 & 0.428 \\ 
    Mip-NeRF360 & 0.159 & 0.354 & \textbf{0.237} & 0.252 \\ 
    Ours-7k & 0.209 & 0.350 & 0.343 & 0.291 \\ 
    Ours-30k & \textbf{0.148} & \textbf{0.218} & 0.244 & \textbf{0.241} \\ 
\end{tabular}
        \label{tab:ttdb_scene_lpips}
\end{table}
\end{appendix}
