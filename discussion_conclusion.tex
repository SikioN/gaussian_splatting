\section{Обсуждение и выводы}

Мы представили первый подход, который действительно позволяет рендеринг радиусных полей в реальном времени с высоким качеством в широком диапазоне сцен и стилей захвата, требуя при этом времени обучения, конкурентного с самыми быстрыми предыдущими методами.

Наш выбор примитива 3D Гауссиан сохраняет свойства объемного рендеринга для оптимизации, одновременно позволяя непосредственно выполнять быстрое растеризованное сплатирование. Наша работа демонстрирует, что — вопреки широко распространенному мнению — непрерывное представление \emph{не} является строго необходимым для обеспечения быстрого и высококачественного обучения радиусных полей.

Большая часть ($\sim$80\%) нашего времени обучения тратится на код Python, поскольку мы создали наше решение на PyTorch, чтобы позволить другим легко использовать наш метод. Только растеризация реализована как оптимизированные CUDA ядра. Мы ожидаем, что перенос всей оптимизации целиком на CUDA, как это сделано, например, в InstantNGP~\cite{mueller2022instant}, может обеспечить значительное дальнейшее ускорение для приложений, где производительность имеет ключевое значение.

Мы также продемонстрировали важность построения на принципах рендеринга в реальном времени, используя мощь GPU и скорость архитектуры программного конвейера растеризации. Эти проектные решения являются ключом к производительности как для обучения, так и для рендеринга в реальном времени, предоставляя конкурентное преимущество в производительности над предыдущим объемным трассированием лучей.

Было бы интересно выяснить, можно ли использовать наши Гауссианы для выполнения реконструкции сетки захваченной сцены. Помимо практических последствий, учитывая широкое использование сеток, это позволило бы нам лучше понять, где именно находится наш метод в континууме между объемными и поверхностными представлениями.

Подводя итог, мы представили первое решение для рендеринга радиусных полей в реальном времени с качеством рендеринга, соответствующим лучшим дорогим предыдущим методам, и временем обучения, конкурентным с самыми быстрыми существующими решениями.