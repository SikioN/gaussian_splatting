\section{Дифференцируемое 3D-сплаттинг Гауссиан}
\label{sec:3d-splats}

Наша цель — оптимизировать представление сцены, которое позволяет синтез высококачественных новых видов, начиная с разреженного набора (SfM) точек без нормалей. Для этого нам нужен примитив, который наследует свойства дифференцируемых объемных представлений, при этом оставаясь неструктурированным и явным для обеспечения очень быстрого рендеринга. Мы выбираем 3D Гауссианы, которые являются дифференцируемыми и могут быть легко спроецированы в 2D-сплаты, что позволяет быстро выполнять $\alpha$-смешивание для рендеринга.

Наше представление имеет сходства с предыдущими методами, использующими 2D-точки ~\cite{yifan19,kopanas21}, и предполагает, что каждая точка является маленьким плоским кругом с нормалью.
Учитывая крайнюю разреженность точек SfM, оценка нормалей крайне затруднена. Аналогично, оптимизация очень зашумленных нормалей из такой оценки также была бы очень сложной задачей.
Вместо этого мы моделируем геометрию как набор 3D Гауссиан, которые не требуют нормалей. Наши Гауссианы определяются полной 3D матрицей ковариации $\Sigma$, заданной в мировом пространстве~\cite{zwicker2001ewa} с центром в точке (среднее значение) $\mu$:
\begin{equation}
    G(x)~= e^{-\frac{1}{2}(x)^{T}\Sigma^{-1}(x)}
\end{equation}
\CORRECTION{где $|\Sigma|$ — это определитель $\Sigma$, являющейся симметричной матрицей размером 3$\times$3}{}. Этот Гауссиан умножается на $\alpha$ в нашем процессе смешивания.

Однако нам нужно спроецировать наши 3D Гауссианы в 2D для рендеринга.
Цвикер и др. ~\shortcite{zwicker2001ewa} демонстрируют, как выполнить эту проекцию в пространство изображения. Учитывая преобразование просмотра $W$,
матрица ковариации $\Sigma'$ в координатах камеры выглядит следующим образом:
\begin{equation}
    \label{eq:volume-render}
    \Sigma' = J W ~\Sigma ~W ^{T}J^{T}
\end{equation}
где $J$ — это якобиан аффинного приближения проективного преобразования. \CORRECTION{Авторы}{Цвикер и др. ~\shortcite{zwicker2001ewa}} также показывают, что если мы пропустим третью строку и столбец $\Sigma'$, 
мы получим матрицу дисперсии размером 2$\times$2 с той же структурой и свойствами, как если бы мы начинали с плоских точек с нормалями, как в предыдущих работах~\cite{kopanas21}.
    
Очевидный подход заключался бы в непосредственной оптимизации матрицы ковариации $\Sigma$ для получения 3D Гауссиан, представляющих радиусное поле. Однако матрицы ковариации имеют физический смысл только тогда, когда они являются положительно полуопределенными. 
Для нашей оптимизации всех параметров мы используем градиентный спуск, который сложно ограничить производством таких допустимых матриц, и шаги обновления и градиенты могут очень легко создавать недопустимые матрицы ковариации. 

В результате мы выбрали более интуитивное, но эквивалентно выразительное представление для оптимизации.
Матрица ковариации $\Sigma$ 3D Гауссиана аналогична описанию конфигурации эллипсоида.
Учитывая матрицу масштабирования $S$ и матрицу поворота $R$, мы можем найти соответствующую $\Sigma$:
\begin{equation}
    \Sigma = RSS^TR^T
\end{equation}

Чтобы позволить независимую оптимизацию обоих факторов, мы храним их отдельно: 3D вектор $s$ для масштабирования и кватернион $q$ для представления вращения. Их можно легко преобразовать в соответствующие матрицы и объединить, убедившись, что $q$ нормализован для получения единичного кватерниона.

Чтобы избежать значительных накладных расходов из-за автоматического дифференцирования во время обучения, мы явно выводим градиенты для всех параметров. Подробности точных вычислений производных находятся в приложении~\ref{sec:appa}.
\CORRECTION{Мы обсудим конкретный случай масштабирования и вращения (т.е., формы) 3D Гауссиан далее.}{}
\CORRECTION{Дифференцируемый рендерер выводит 
матрицу ковариации экранного пространства $\Sigma'$ размером $2\times2$.
Учитывая градиент потерь $\frac{dL}{d\Sigma'}$ %
(который мы считаем выходным сигналом дифференцируемого рендерера), 
мы можем применить цепное правило для вычисления градиентов и распространения потерь.}{}

Это представление анизотропной ковариации — подходящее для оптимизации — позволяет нам оптимизировать 3D Гауссианы для адаптации к геометрии различных форм в захваченных сценах, что приводит к довольно компактному представлению. 
Рис.~\ref{fig:aniso-cov} иллюстрирует такие случаи.
\begin{figure}[!h]
    \begin{overpic}[width=\columnwidth]{figures/anisotropic/real2}
        \put (61,3) {\color{white}Оригинал}
        \put (82.2,5) {\color{white}Сжатые} 
        \put (82,1.2) {\color{white}Гауссианы}
    \end{overpic}
    
    \caption{
        Мы визуализируем 3D Гауссианы после оптимизации, сжимая их на 60\% (справа). Это четко показывает анизотропные формы 3D Гауссиан, которые компактно представляют сложную геометрию после оптимизации. Слева — фактическое отрендеренное изображение.
    }
    \label{fig:aniso-cov}
\end{figure}